\documentclass[a4paper,11pt]{report}
\usepackage[top=2cm, bottom=2cm, left=2.5cm, right=2.5cm]{geometry}
\usepackage[utf8]{inputenc}
\usepackage{cmap}
\usepackage{graphicx} 
\usepackage[T2A]{fontenc}
\usepackage[english,russian]{babel}
\usepackage{amsmath}
\usepackage{amsfonts}
\usepackage{listings}
\usepackage{multirow}
\usepackage{hhline}
\usepackage{float}

% Title Page
\title{Отчёт по практическому заданию №1 ``Прямые и итерационные методы решения СЛАУ''}
\author{Маслов Н.С. группа 205}

\lstset{
language=C,
breaklines=true,
basicstyle=\small\ttfamily,
numbers=left,
numberstyle=\tiny,
frame=tb,
columns=fullflexible,
showstringspaces=false,
extendedchars=\true,
keepspaces=true
}

\floatstyle{ruled}
\newfloat{bash}{H}{lop}
\floatname{bash}{Консоль}

\begin{document}
\begin{titlepage}

\newgeometry{top=2.5cm, left=2.5cm, right=2.5cm}



\center % Center everything on the page
 
%----------------------------------------------------------------------------------------
%	HEADING SECTIONS
%----------------------------------------------------------------------------------------

\textsc{\Large Московский государственный университет им. М.В.Ломоносова}\\[0.5cm] % Name of your university/college
\textsc{\Large Факультет вычислительной математики и кибернетики}\\[7.5cm] % Major heading such as course name

%----------------------------------------------------------------------------------------
%	TITLE SECTION
%----------------------------------------------------------------------------------------

\textsc{\LARGE Отчёт о выполнении практической работы № 1}\\[1cm]
\textsc{\LARGE \huge \bfseries Прямые и итерационные методы решения СЛАУ}\\[5cm] % Title of your document
 
%----------------------------------------------------------------------------------------
%	AUTHOR SECTION
%----------------------------------------------------------------------------------------

\begin{minipage}{0.4\textwidth}
\begin{flushleft} \large

\end{flushleft}
\end{minipage}
~
\begin{minipage}{0.4\textwidth}
\begin{flushright} \large
Выполнил \\[0.3cm]
Маслов Н.C. \\ группа 205 % Your name
\end{flushright}
\end{minipage}\\[4cm]

% If you don't want a supervisor, uncomment the two lines below and remove the section above
%\Large \emph{Author:}\\
%John \textsc{Smith}\\[3cm] % Your name

%----------------------------------------------------------------------------------------
%	DATE SECTION
%----------------------------------------------------------------------------------------

{\large Москва, 2014}\\[3cm] % Date, change the \today to a set date if you want to be precise

%----------------------------------------------------------------------------------------
%	LOGO SECTION
%----------------------------------------------------------------------------------------

%\includegraphics{Logo}\\[1cm] % Include a department/university logo - this will require the graphicx package
 
%----------------------------------------------------------------------------------------

\vfill % Fill the rest of the page with whitespace

\end{titlepage}

%\tableofcontents

\section*{Постановка задачи}
Дана система уравнений $Ax=f$ порядка $n\times n$ с невырожденной матрицей $A$. Требуется написать
программу, рещающую систему линейных алгебраических уравнений заданного пользователем размера ($n$ - параметр программы)
методом Гаусса и методом Гаусса с выбором главного элемента.

Предусмотреть возможность задания элементов матрицы системы и её правой части как во входном файле данных, так и путём
задания специальных формул.

\section*{Цели и задачи практической работы}
\begin{enumerate}
 \item Решить заданную СЛАУ методом Гаусса и методом Гаусса с выбором главного элемента;
 \item Вычислить определитель матрицы $det(A)$;
 \item Вычислить обратную матрицу $A^{-1}$;
 \item Исследовать вопрос вычислительной устойчивости метода Гаусса (при больших значениях параметра n);
 \item Правильность решения СЛАУ подтвердить системой тестов (например, можно использовать ресурсы on-line
 системы Wolfram|Alpha, пакета Maple и т.п.)
\end{enumerate}

\section*{Описание методов решения}
\subsection*{Метод Гаусса}
Метод Гаусса построен на идее приведения элементарными преобразованиями матрицы к диагональному виду, при этом параллельно
те же преобразования делаются с вектором правой части выражения.

Ниже приведён алгоритм решения и исходный код модуля. (Исходный код базовых операций с векторами и матрицами приведён
в приложении 1).

Прямой ход метода Гаусса:
\begin{enumerate}
 \item Положим $i = 1$;
 \item Выберем первую строку матрицы, в которой $i$-й элемент ненулевой; если это не $i$-я строка, то поменяем текущую и найденную
 строку местами (то же преобразование проведём с вектором правой части);
 \item Разделим эту строку (и соответствующее значение в векторе правой части) на значение $i$-го элемента: так на диагонали мы 
 получим единицу;
 \item Вычтем из последующих строк, в который $i$-й элемент ненулевой, текущую, домноженную на значение $i$-го элемента уменьшаемой
 строки: так мы получим нулевые элементы в $i$-м столбце на диагонали. Аналогичное действие проделываем в векторе правой части;
 \item Увеличим $i$ на 1 и, если $i \neq n$, перейдём к шагу 2.
\end{enumerate}

В результате прямого хода получается треугольная матрица с единицами на диагонали. При обратном ходе метода Гаусса матрица переходит к
диагональному виду. Обратный ход метода Гаусса проводится следующим образом:
\begin{enumerate}
 \item Положим $i = n$, где $n$ - порядок матрицы (выбираем последнюю строку);
 \item Вычтем из всех вышестоящих строк $i$-ю, домноженную на значение соответствующего элемента строки (задача - оставить нули в 
 $i$-м столбце над диагональю); аналогичное действие делаем с вектором правой части;
 \item Уменьшаем $i$ на 1; если $i \neq 0$, переходим к шагу 2.
\end{enumerate}


Рассмотрим пример решения СЛАУ методом Гаусса.

Дана система уравнений

\[
\begin{cases}
 x + 2y + 3z = 4 \\
 4x + 5z = 8\\
 2z = 0
\end{cases}
\]

~\\
Решение методом Гаусса:

$
\left[
\begin{array}{ccc|c}
 1 & 2 & 3 & 4 \\
 4 & 0 & 5 & 8 \\
 0 & 0 & 2 & 0 
\end{array}
\right] \rightarrow
$
$
\left[
\begin{array}{ccc|c}
 1 & 2 & 3 & 4 \\
 0 & -8 & -7 & -8 \\
 0 & 0 & 2 & 0 
\end{array}
\right] \rightarrow
$
$
\left[
\begin{array}{ccc|c}
 1 & 2 & 3 & 4 \\
 0 & 1 & 7/8 & 1 \\
 0 & 0 & 1 & 0 
\end{array}
\right] \rightarrow
$
$
\left[
\begin{array}{ccc|c}
 1 & 2 & 0 & 4 \\
 0 & 1 & 0 & 1 \\
 0 & 0 & 1 & 0 
\end{array}
\right] \rightarrow
$


$
\left[
\begin{array}{ccc|c}
 1 & 0 & 0 & 2 \\
 0 & 1 & 0 & 1 \\
 0 & 0 & 1 & 0 
\end{array}
\right]
$

~\\
Ответ: $(2, 1, 0)$.

Аналогичным образом метод Гаусса позволяет искать обратную матрицу; в этом случае в правой части записи вместо вектора правой
части записывается единичная матрица соответствующего размера.

В коде программы был применён унифицированный подход: к реализации алгоритма метода Гаусса подключается набор функций-операторов
правой части, таким образом, можно одним и тем же кодом работать и с вектором для решения СЛАУ, и с матрицей для поиска обратной.

Исходный код модуля на языке Си:

\lstinputlisting[caption={Исходный код метода Гаусса (файл gauss.c)}]{../gauss.c}

\subsection*{Модифицированный метод Гаусса}
Отличие модифицированного метода Гаусса от классического заключается в том, что на каждом шаге прямого хода мы ищем наибольший по
модулю элемент в строке, а не среди строк в одном столбце. С тем учётом, что найденный элемент будет наибольшим по модулю, после
деления строки на это значение мы получим значения в интервале $[0, 1]$, что позволяет несколько сохранить точность вычислений с 
учётом ошибок округления (из-за конечности длины машинного слова).

Поскольку в этом методе происходит смена мест столбцов, а не строк (что равнозначно переименованию переменных), появляется
необходимость переставить элементы в векторе результата (соответственно, строки в матрице, если мы считаем этим методом обратную
матрицу).

Исходный код модуля на языке Си:
~\\

\lstinputlisting[caption={Исходный код модифицированного метода Гаусса (файл gauss\_mod.c)}]{../gauss_mod.c}


\subsection*{Метод верхней релаксации, метод Зейделя}
Метод верхней релаксации и метод Зейделя (как частный случай метода релаксаци) - итерационные методы решения СЛАУ. Алгоритм 
работы заключается в следующем.

Рассмотрим квадратную матрицу 

$$
A = \begin{bmatrix}
 a_{11} & a_{12} & \hdots & a_{1n} \\
 a_{21} & a_{22} & \hdots & a_{2n} \\
 \vdots & \vdots & \ddots & \vdots \\
 a_{n1} & a_{n2} & \hdots & a_{nn}
\end{bmatrix}
$$

Разложим её на сумму трёх матриц

$$A = D + T_H + T_B$$,

где $D$ - часть матрицы $A$, содержащая её главную диагональ, $T_H$ - нижняя треугольная часть, $T_B$ - верхняя треугольная часть
(без диагонали).

Введём параметр $\omega$ и запишем рекуррентное соотношение

$$(D + \omega T_H)\frac{(x_{k+1} - x_k)}{\omega} + Ax_k = f$$

Если матрица $A$ самосопряжённая и положительно определённая, либо является матрицей с диагональным преобладанием (при $\omega = 1$),
при итерировании алгоритма по $k$ вектор $x_k$ будет сходиться к решению СЛАУ $Ax = f$.

При значении параметра $\omega = 1$ метод называется \textit{методом Зейделя}.

Для построения алгоритма вычисления очередной итерации нужно разделить в левой части члены, содержащие $x_k$ и $x_{k+1}$:

$$\bigg(\frac{1}{\omega}D + T_H\bigg)x_{k+1} + \Bigg[\bigg(1 - \frac{1}{\omega}\bigg)D + T_B\Bigg]x_k = f$$

При переходе от векторной записи к поэлементной, получаем формулу для компонент $x_{k+1}$:

$$x_i^{k+1} = x_i^k + \frac{\omega}{a_{ii}}\Bigg(f_i - \sum\limits_{j=1}^{i-1}a_{ij}x_j^{k+1} - \sum\limits_{j=i}^na_{ij}x_j^k\Bigg), i = 1,...,n$$

Эта формула уже довольно просто реализуется в алгоритме.

Условием окончания итерирования можно установить определённое число итераций, либо достижения достаточной точности решения. Последнее
можно определить, наблюдая за нормой вектора невязки $\|Ax_k - f\|$. В приведённой ниже реализации применён комбинированный подход:
ограничено число итераций, а также ведётся контроль нормы невязки.

Исходный код модуля на языке Си:

\lstinputlisting[caption={Метод верхней релаксации (файл relax.c)}]{../relax.c}

\section*{Тестирование}
Для тестирования в программу был добавлен вывод значения невязки ответа, а для метода релаксации - вывод значения невязки результата
на каждой итерации.

\subsection*{Тест 1. Единичная матрица}
Отчёт о тестировании каждого алгоритма с единичной матрицей размера $3\times3$ и вектором правой части $(1, 2, 3)^T$:

\begin{bash}
\caption{Метод Гаусса, решение СЛАУ}
\begin{verbatim}
 $ ./matrix -m gauss -o solve < tests/solve1.txt
 [INPUT] Type N, than matrix (NxN)
 [INPUT] Type vector f (length 3)
 [OUTPUT] Result
 1 2 3 
 [OUTPUT] Residual: 0
\end{verbatim}
\end{bash}

\begin{bash}
 \caption{Метод Гаусса, подсчёт определителя}
 \begin{verbatim}
  $ ./matrix -m gauss -o det < tests/solve1.txt 
[INPUT] Type N, than matrix (NxN)
[OUTPUT] Result
1
 \end{verbatim}
\end{bash}

\begin{bash}
 \caption{Метод Гаусса, подсчёт обратной матрицы}
 \begin{verbatim}
  $ ./matrix -m gauss -o invert < tests/solve1.txt 
[INPUT] Type N, than matrix (NxN)
[OUTPUT] Result
1 0 0 
0 1 0 
0 0 1 
 \end{verbatim}
\end{bash}


\begin{bash}
\caption{Модифицированный метод Гаусса, решение СЛАУ}
\begin{verbatim}
 $ ./matrix -m gauss_mod -o solve < tests/solve1.txt 
[INPUT] Type N, than matrix (NxN)
[INPUT] Type vector f (length 3)
[OUTPUT] Result
1 2 3 
[OUTPUT] Residual: 0
\end{verbatim}
\end{bash}
\begin{bash}
 \caption{Модифицированный метод Гаусса, подсчёт определителя}
 \begin{verbatim}
  $ ./matrix -m gauss_mod -o det < tests/solve1.txt 
[INPUT] Type N, than matrix (NxN)
[OUTPUT] Result
1
 \end{verbatim}
\end{bash}

\begin{bash}
 \caption{Модифицированный метод Гаусса, подсчёт обратной матрицы}
 \begin{verbatim}
  $ ./matrix -m gauss_mod -o invert < tests/solve1.txt 
[INPUT] Type N, than matrix (NxN)
[OUTPUT] Result
1 0 0 
0 1 0 
0 0 1 
 \end{verbatim}
\end{bash}

\begin{bash}
\caption{Метод верхней релаксации, решение СЛАУ, $\omega = 1$, $\varepsilon = 0.00001$}
\begin{verbatim}
$ ./matrix -m relax -o solve < tests/solve1.txt 
[INPUT] Type N, than matrix (NxN)
[INPUT] Type vector f (length 3)
[INPUT] Type 'omega' for overrelaxation method
[INPUT] Type precisioncoefficient (eps)
[RELAX] Iteration 1, residual 0
[OUTPUT] Result
1 2 3 
[OUTPUT] Residual: 0
\end{verbatim}
\end{bash}

Как видно, все три алгоритма справляются с задачами с максимальной точностью.

\subsection*{Тест 2. СЛАУ из списка примеров}

В списке примеров (вариант 12) предложена следующая СЛАУ с квадратной невырожденной матрицей:

$$
\begin{cases}
 2x_1 - 2x_2 + x_4 = -3,\\
 2x_1 + 3x_2 + x_3 - 3x_4 = -6,\\
 3x_1 + 4x_2 - x_3 + 2x_4 = 0,\\
 x_1 + 3x_2 + x_3 - x_4 = 2
\end{cases}
$$


\begin{bash}
\caption{Метод Гаусса, решение СЛАУ}
\begin{verbatim}
$ ./matrix -m gauss -o solve < tests/solve2.txt 
[INPUT] Type N, than matrix (NxN)
[INPUT] Type vector f (length 4)
[OUTPUT] Result
-2 1 4 3 
[OUTPUT] Residual: 1.5384e-15
\end{verbatim}
\end{bash}

\begin{bash}
 \caption{Метод Гаусса, подсчёт определителя}
 \begin{verbatim}
$ ./matrix -m gauss -o det < tests/solve2.txt 
[INPUT] Type N, than matrix (NxN)
[OUTPUT] Result
-53
 \end{verbatim}
\end{bash}

\begin{bash}
 \caption{Метод Гаусса, подсчёт обратной матрицы}
 \begin{verbatim}
$ ./matrix -m gauss -o invert -f latex < tests/solve2.txt 
[INPUT] Type N, than matrix (NxN)
[OUTPUT] Result
 \end{verbatim}
$$\begin{pmatrix}
0.26415  & 0.16981  & 0.075472  & -0.09434  \\
-0.16981  & -0.037736  & 0.09434  & 0.13208  \\
0.37736  & -0.4717  & -0.32075  & 1.1509  \\
0.13208  & -0.41509  & 0.037736  & 0.45283 
\end{pmatrix}$$
\end{bash}


\begin{bash}
\caption{Модифицированный метод Гаусса, решение СЛАУ}
\begin{verbatim}
 $ ./matrix -m gauss_mod -o solve < tests/solve2.txt 
[INPUT] Type N, than matrix (NxN)
[INPUT] Type vector f (length 4)
[OUTPUT] Result
-2 1 4 3 
[OUTPUT] Residual: 4.3512e-15
\end{verbatim}
\end{bash}

\begin{bash}
 \caption{Модифицированный метод Гаусса, подсчёт определителя}
 \begin{verbatim}
$ ./matrix -m gauss_mod -o det < tests/solve2.txt 
[INPUT] Type N, than matrix (NxN)
[OUTPUT] Result
-53
 \end{verbatim}
\end{bash}

\begin{bash}
 \caption{Модифицированный метод Гаусса, подсчёт обратной матрицы}
 \begin{verbatim}
$ ./matrix -m gauss_mod -o invert -f latex < tests/solve2.txt 
[INPUT] Type N, than matrix (NxN)
[OUTPUT] Result
 \end{verbatim}
$$\begin{pmatrix}
0.26415  & 0.16981  & 0.075472  & -0.09434  \\
-0.16981  & -0.037736  & 0.09434  & 0.13208  \\
0.37736  & -0.4717  & -0.32075  & 1.1509  \\
0.13208  & -0.41509  & 0.037736  & 0.45283 
\end{pmatrix}$$
\end{bash}

\begin{bash}
\caption{Метод верхней релаксации, решение СЛАУ, $\omega = 1$, $\varepsilon = 0.00001$}
\begin{verbatim}
$ ./matrix -m relax -o solve < tests/solve2.txt 
[INPUT] Type N, than matrix (NxN)
[INPUT] Type vector f (length 4)
[INPUT] Type 'omega' for overrelaxation method
[INPUT] Type precisioncoefficient (eps)
[RELAX] Iteration 1, residual 49.003
[RELAX] Iteration 2, residual 379.64
...
[RELAX] Iteration 50, residual 1.6483e+45
[RELAX] Iteration 51, residual 1.2745e+46
[OUTPUT] Result
2.1895e+44 -6.1365e+44 -2.991e+45 -4.613e+45 
[OUTPUT] Residual: 1.2745e+46
\end{verbatim}
\end{bash}

Как видно из отчёта, такая система не решается методом верхней релаксации, так как матрица коэффициентов не является положительно
определённой или матрицей с диагональным преобладанием. Вектор решения в этом случае должен расходиться, что мы и наблюдаем.
Однако, при решении системы методом Гаусса мы получаем достаточно точное решение (порядок нормы невязки $10^{-15}$).

\subsection*{Тест 3. Положительно определённая матрица (для демонстрации итерационного метода)}

Для демонстрации сходимости метода верхней релаксации, рассмотрим положительно определённую матрицу
$$\begin{pmatrix}
    1 & 1 & 0 \\
    1 & 2 & 0 \\
    0 & 0 & 1
\end{pmatrix}$$

с вектором правой части $(1, 2, 3)$.

\begin{bash}
\caption{Метод Гаусса, решение СЛАУ}
\begin{verbatim}
$ ./matrix -m gauss -o solve < tests/solve_relax2.txt 
[INPUT] Type N, than matrix (NxN)
[INPUT] Type vector f (length 3)
[OUTPUT] Result
0 1 3 
[OUTPUT] Residual: 0
\end{verbatim}
\end{bash}


\begin{bash}
\caption{Метод верхней релаксации, решение СЛАУ, $\omega = 1$, $\varepsilon = 0.00001$}
\begin{verbatim}
$ ./matrix -m relax -o solve < tests/solve_relax2.txt 
[INPUT] Type N, than matrix (NxN)
[INPUT] Type vector f (length 3)
[INPUT] Type 'omega' for overrelaxation method
[INPUT] Type precisioncoefficient (eps)
[RELAX] Iteration 1, residual 0.5
[RELAX] Iteration 2, residual 0.25
[RELAX] Iteration 3, residual 0.125
[RELAX] Iteration 4, residual 0.0625
[RELAX] Iteration 5, residual 0.03125
[RELAX] Iteration 6, residual 0.015625
[RELAX] Iteration 7, residual 0.0078125
[RELAX] Iteration 8, residual 0.0039062
[RELAX] Iteration 9, residual 0.0019531
[RELAX] Iteration 10, residual 0.00097656
[RELAX] Iteration 11, residual 0.00048828
[RELAX] Iteration 12, residual 0.00024414
[RELAX] Iteration 13, residual 0.00012207
[RELAX] Iteration 14, residual 6.1035e-05
[RELAX] Iteration 15, residual 3.0518e-05
[RELAX] Iteration 16, residual 1.5259e-05
[RELAX] Iteration 17, residual 7.6294e-06
[OUTPUT] Result
1.5259e-05 0.99999 3 
[OUTPUT] Residual: 7.6294e-06
\end{verbatim}
\end{bash}

\begin{bash}
\caption{Метод верхней релаксации, решение СЛАУ, $\omega = 0.8$, $\varepsilon = 0.00001$}
\begin{verbatim}
$ ./matrix -m relax -o solve < tests/solve_relax2.txt 
[INPUT] Type N, than matrix (NxN)
[INPUT] Type vector f (length 3)
[INPUT] Type 'omega' for overrelaxation method
[INPUT] Type precisioncoefficient (eps)
[RELAX] Iteration 1, residual 0.74833
[RELAX] Iteration 2, residual 0.31496
[RELAX] Iteration 3, residual 0.20207
[RELAX] Iteration 4, residual 0.12653
[RELAX] Iteration 5, residual 0.077129
[RELAX] Iteration 6, residual 0.046529
[RELAX] Iteration 7, residual 0.027968
[RELAX] Iteration 8, residual 0.016791
[RELAX] Iteration 9, residual 0.010077
[RELAX] Iteration 10, residual 0.0060464
[RELAX] Iteration 11, residual 0.0036279
[RELAX] Iteration 12, residual 0.0021768
[RELAX] Iteration 13, residual 0.0013061
[RELAX] Iteration 14, residual 0.00078364
[RELAX] Iteration 15, residual 0.00047018
[RELAX] Iteration 16, residual 0.00028211
[RELAX] Iteration 17, residual 0.00016927
[RELAX] Iteration 18, residual 0.00010156
[RELAX] Iteration 19, residual 6.0936e-05
[RELAX] Iteration 20, residual 3.6562e-05
[RELAX] Iteration 21, residual 2.1937e-05
[RELAX] Iteration 22, residual 1.3162e-05
[RELAX] Iteration 23, residual 7.8973e-06
[OUTPUT] Result
1.5795e-05 0.99999 3 
[OUTPUT] Residual: 7.8973e-06
\end{verbatim}
\end{bash}

\begin{bash}
\caption{Метод верхней релаксации, решение СЛАУ, $\omega = 1.3$, $\varepsilon = 0.00001$}
\begin{verbatim}
$ ./matrix -m relax -o solve < tests/solve_relax2.txt 
[INPUT] Type N, than matrix (NxN)
[INPUT] Type vector f (length 3)
[INPUT] Type 'omega' for overrelaxation method
[INPUT] Type precisioncoefficient (eps)
[RELAX] Iteration 1, residual 1.4396
[RELAX] Iteration 2, residual 0.32955
[RELAX] Iteration 3, residual 0.13734
[RELAX] Iteration 4, residual 0.029229
[RELAX] Iteration 5, residual 0.01338
[RELAX] Iteration 6, residual 0.0026558
[RELAX] Iteration 7, residual 0.0013379
[RELAX] Iteration 8, residual 0.00025454
[RELAX] Iteration 9, residual 0.00013785
[RELAX] Iteration 10, residual 2.6498e-05
[RELAX] Iteration 11, residual 1.4662e-05
[RELAX] Iteration 12, residual 3.0062e-06
[OUTPUT] Result
5.6956e-06 1 3 
[OUTPUT] Residual: 3.0062e-06
\end{verbatim}
\end{bash}

\begin{bash}
\caption{Метод верхней релаксации, решение СЛАУ, $\omega = 1.6$, $\varepsilon = 0.00001$}
\begin{verbatim}
$ ./matrix -m relax -o solve < tests/solve_relax2.txt 
[INPUT] Type N, than matrix (NxN)
[INPUT] Type vector f (length 3)
[INPUT] Type 'omega' for overrelaxation method
[INPUT] Type precisioncoefficient (eps)
[RELAX] Iteration 1, residual 2.5768
[RELAX] Iteration 2, residual 1.4653
[RELAX] Iteration 3, residual 0.8945
[RELAX] Iteration 4, residual 0.53358
[RELAX] Iteration 5, residual 0.32077
[RELAX] Iteration 6, residual 0.19234
[RELAX] Iteration 7, residual 0.11543
[RELAX] Iteration 8, residual 0.069251
[RELAX] Iteration 9, residual 0.041552
[RELAX] Iteration 10, residual 0.024931
[RELAX] Iteration 11, residual 0.014959
[RELAX] Iteration 12, residual 0.0089751
[RELAX] Iteration 13, residual 0.0053851
[RELAX] Iteration 14, residual 0.003231
[RELAX] Iteration 15, residual 0.0019386
[RELAX] Iteration 16, residual 0.0011632
[RELAX] Iteration 17, residual 0.0006979
[RELAX] Iteration 18, residual 0.00041874
[RELAX] Iteration 19, residual 0.00025125
[RELAX] Iteration 20, residual 0.00015075
[RELAX] Iteration 21, residual 9.0448e-05
[RELAX] Iteration 22, residual 5.4269e-05
[RELAX] Iteration 23, residual 3.2561e-05
[RELAX] Iteration 24, residual 1.9537e-05
[RELAX] Iteration 25, residual 1.1722e-05
[RELAX] Iteration 26, residual 7.0333e-06
[OUTPUT] Result
-3.4116e-06 1 3 
[OUTPUT] Residual: 7.0333e-06
\end{verbatim}
\end{bash}

Как видно из отчёта, метод релаксации справляется с задачей с достаточной точностью, при этом наименьшее число итераций для 
достижения необходимой точности мы получили при значении параметра $\omega = 1.3$. (Значение было подобрано эмпирически, возможно,
есть более оптимальное значение, но данных примеров достаточно для демонстрации особенностей использования метода релаксации).

\subsection*{Тест 4. Матрица, заданная по формуле (1)}

Пусть матрица и вектор правой части заданы с помощью формул:

$$A_{ij} = \begin{cases}
       \frac{i + j}{m + n}, i \neq j, \\
       n + m^2 + \frac{j}{m} + \frac{i}{n}, i = j
      \end{cases}
$$,
$$ b_i = 200 + 50i $$

где $i, j = 1,...,n, n = 20, m = 8$.

Матрицы заполняются внутри программы для того, чтобы не терять точность при переводе чисел из формата с плавающей точкой в
текстовый и обратно.

Вычисление обратной матрицы здесь опущено из-за размеров матрицы (20).

\begin{bash}
\caption{Метод Гаусса, решение СЛАУ}
\begin{verbatim}
$ ./matrix -m gauss -o solve -i formula1 
[OUTPUT] Result
2.1322 2.6631 3.1927 3.721 4.248 4.7737 5.2981 5.8213 6.3431 6.8637 7.3831 7.9012 
8.418 8.9335 9.4478 9.9609 10.473 10.983 11.493 12.001 
[OUTPUT] Residual: 1.033e-12
\end{verbatim}
\end{bash}

\begin{bash}
 \caption{Метод Гаусса, подсчёт определителя}
 \begin{verbatim}
$ ./matrix -m gauss -o det -i formula1 
[OUTPUT] Result
4.6398e+38
 \end{verbatim}
\end{bash}

\begin{bash}
\caption{Модифицированный метод Гаусса, решение СЛАУ}
\begin{verbatim}
$ ./matrix -m gauss_mod -o solve -i formula1 
[OUTPUT] Result
2.1322 2.6631 3.1927 3.721 4.248 4.7737 5.2981 5.8213 6.3431 6.8637 7.3831 7.9012
8.418 8.9335 9.4478 9.9609 10.473 10.983 11.493 12.001 
[OUTPUT] Residual: 1.033e-12
\end{verbatim}
\end{bash}

\begin{bash}
 \caption{Модифицированный метод Гаусса, подсчёт определителя}
 \begin{verbatim}
$ ./matrix -m gauss_mod -o det -i formula1 
[OUTPUT] Result
4.6398e+38
 \end{verbatim}
\end{bash}

\begin{bash}
\caption{Метод верхней релаксации, решение СЛАУ, $\omega = 1$, $\varepsilon = 0.00001$}
\begin{verbatim}
$ ./matrix -m relax -o solve -i formula1 
[INPUT] Type 'omega' for overrelaxation method
1
[INPUT] Type precisioncoefficient (eps)
0.00001
[RELAX] Iteration 1, residual 345.83
[RELAX] Iteration 2, residual 17.645
[RELAX] Iteration 3, residual 0.45965
[RELAX] Iteration 4, residual 0.0084147
[RELAX] Iteration 5, residual 0.00053034
[RELAX] Iteration 6, residual 1.4704e-05
[RELAX] Iteration 7, residual 2.7193e-07
[OUTPUT] Result
2.1322 2.6631 3.1927 3.721 4.248 4.7737 5.2981 5.8213 6.3431 6.8637 7.3831 7.9012
8.418 8.9335 9.4478 9.9609 10.473 10.983 11.493 12.001 
[OUTPUT] Residual: 2.7193e-07
\end{verbatim}
\end{bash}

\begin{bash}
\caption{Метод верхней релаксации, решение СЛАУ, $\omega = 0.8$, $\varepsilon = 0.00001$}
\begin{verbatim}
$ ./matrix -m relax -o solve -i formula1 
[INPUT] Type 'omega' for overrelaxation method
0.8
[INPUT] Type precisioncoefficient (eps)
0.00001
[RELAX] Iteration 1, residual 516.51
[RELAX] Iteration 2, residual 86.73
[RELAX] Iteration 3, residual 17.223
[RELAX] Iteration 4, residual 3.7086
[RELAX] Iteration 5, residual 0.79942
[RELAX] Iteration 6, residual 0.16788
[RELAX] Iteration 7, residual 0.034243
[RELAX] Iteration 8, residual 0.00681
[RELAX] Iteration 9, residual 0.0013281
[RELAX] Iteration 10, residual 0.00025572
[RELAX] Iteration 11, residual 4.8997e-05
[RELAX] Iteration 12, residual 9.424e-06
[OUTPUT] Result
2.1322 2.6631 3.1927 3.721 4.248 4.7737 5.2981 5.8213 6.3431 6.8637 7.3831 7.9012
8.418 8.9335 9.4478 9.9609 10.473 10.983 11.493 12.001 
[OUTPUT] Residual: 9.424e-06
\end{verbatim}
\end{bash}

\begin{bash}
\caption{Метод верхней релаксации, решение СЛАУ, $\omega = 0.8$, $\varepsilon = 0.00001$}
\begin{verbatim}
$ ./matrix -m relax -o solve -i formula1 
[INPUT] Type 'omega' for overrelaxation method
1.3
[INPUT] Type precisioncoefficient (eps)
0.00001
[RELAX] Iteration 1, residual 1421.7
[RELAX] Iteration 2, residual 572.57
[RELAX] Iteration 3, residual 225.95
[RELAX] Iteration 4, residual 87.296
[RELAX] Iteration 5, residual 33.075
[RELAX] Iteration 6, residual 12.316
[RELAX] Iteration 7, residual 4.5156
[RELAX] Iteration 8, residual 1.6332
[RELAX] Iteration 9, residual 0.58347
[RELAX] Iteration 10, residual 0.20615
[RELAX] Iteration 11, residual 0.072096
[RELAX] Iteration 12, residual 0.024978
[RELAX] Iteration 13, residual 0.0085782
[RELAX] Iteration 14, residual 0.0029218
[RELAX] Iteration 15, residual 0.00098738
[RELAX] Iteration 16, residual 0.00033118
[RELAX] Iteration 17, residual 0.00011028
[RELAX] Iteration 18, residual 3.6462e-05
[RELAX] Iteration 19, residual 1.1973e-05
[RELAX] Iteration 20, residual 3.9048e-06
[OUTPUT] Result
2.1322 2.6631 3.1927 3.721 4.248 4.7737 5.2981 5.8213 6.3431 6.8637 7.3831 7.9012
8.418 8.9335 9.4478 9.9609 10.473 10.983 11.493 12.001 
[OUTPUT] Residual: 3.9048e-06
\end{verbatim}
\end{bash}

Все три метода справились с задачей с достаточной точностью.

\subsection*{Тест 5. Матрица, заданная по формуле (2)}

Пусть матрица и вектор правой части заданы по формуле:

$$A_{ij} = \begin{cases}
       q_M^{i + j} + 0.1(j - i), i \neq j, \\
       (q_M - 1)^{i + j}, i = j
      \end{cases}
$$,
$$ q_M = 1.001 - 2 \cdot 10^{-3} \cdot M $$
$$ b_i = n \cdot \exp \Big(\frac{x}{i}\Big) \cdot \cos(x) $$

где $i, j = 1,...,n, n = 100, M = 4$.

Матрицы заполняются внутри программы для того, чтобы не терять точность при переводе чисел из формата с плавающей точкой в
текстовый и обратно.

Вычисление обратной матрицы здесь опущено из-за размеров матрицы (100).


\begin{bash}
 \caption{Метод Гаусса, подсчёт определителя}
 \begin{verbatim}
$ ./matrix -m gauss -o det -i formula2 
[OUTPUT] Result
5.4745e-26
 \end{verbatim}
\end{bash}


\begin{bash}
 \caption{Модифицированный метод Гаусса, подсчёт определителя}
 \begin{verbatim}
$ ./matrix -m gauss_mod -o det -i formula2 
[OUTPUT] Result
5.4745e-26
 \end{verbatim}
\end{bash}

Положим $x = 0$. Вывод результата будет опущен, для оценки работы алгоритмов достаточно значения нормы невязки.

\begin{bash}
\caption{Метод Гаусса, решение СЛАУ ($x = 0$)}
\begin{verbatim}
$ ./matrix -m gauss -o solve -i formula2 
[INPUT] Type X:
0
[OUTPUT] Result: ... (опущено, очень длинный вывод)
[OUTPUT] Residual: 1.1188e-08
\end{verbatim}
\end{bash}


\begin{bash}
\caption{Модифицированный метод Гаусса, решение СЛАУ ($x = 0$)}
\begin{verbatim}
$ ./matrix -o solve -m gauss_mod -i formula2
[INPUT] Type X:
0
[OUTPUT] Result: ... (опущено, очень длинный вывод)
[OUTPUT] Residual: 2.67e-13
\end{verbatim}
\end{bash}

Как мы видим, модифицированный метод Гаусса дал куда лучшее решение, чем традиционный.

Метод верхней релаксации не подходит для решения этой задачи, так как не будет сходимости вектора решения.

Рассмотрим также решения при других значениях $x$.

Положим $x = -10$.

\begin{bash}
\caption{Метод Гаусса, решение СЛАУ ($x = -10$)}
\begin{verbatim}
$ ./matrix -m gauss -o solve -i formula2 
[INPUT] Type X:
-10
[OUTPUT] Result: ... (опущено, очень длинный вывод)
[OUTPUT] Residual: 1.0741e-06
\end{verbatim}
\end{bash}


\begin{bash}
\caption{Модифицированный метод Гаусса, решение СЛАУ ($x = -10$)}
\begin{verbatim}
$ ./matrix -m gauss_mod -o solve -i formula2 
[INPUT] Type X:
-10
[OUTPUT] Result: ... (опущено, очень длинный вывод)
[OUTPUT] Residual: 7.4998e-12
\end{verbatim}
\end{bash}

Положим $x = 10$.

\begin{bash}
\caption{Метод Гаусса, решение СЛАУ ($x = 10$)}
\begin{verbatim}
$ ./matrix -m gauss -o solve -i formula2 
[INPUT] Type X:
10
[OUTPUT] Result: ... (опущено, очень длинный вывод)
[OUTPUT] Residual: 0.007408
\end{verbatim}
\end{bash}


\begin{bash}
\caption{Модифицированный метод Гаусса, решение СЛАУ ($x = 10$)}
\begin{verbatim}
$ ./matrix -m gauss_mod -o solve -i formula2 
[INPUT] Type X:
10
[OUTPUT] Result: ... (опущено, очень длинный вывод)
[OUTPUT] Residual: 6.242e-08
\end{verbatim}
\end{bash}

Как видно, при увеличении по модулю значения $x$, точность теряется у обоих методов решения, но при этом
модифицированный метод Гаусса даёт гораздо более высокую точность решения.

\section*{Выводы}
Из проделанной работы можно сделать следующие основные выводы.

У каждого из описанных выше методов есть свои достоинства и недостатки в своих сферах применения. 

Метод Гаусса является достаточно универсальным методом решения СЛАУ, при этом предоставляя возможность вычисления определителя 
матрицы и подсчёта обратной матрицы. Модификация метода Гаусса с выбором главного элемента в строке в определённом круге задач 
может значительно уточнить решение.

Итерационные методы позволяют достаточно быстро получить результат требуемой точности, но требуют
определённой подготовки матрицы для сходимости. Итерационные методы, как правило, более устойчивы к особенностям хранения
действительных чисел в машинном слове и к соответствующим ошибкам при вычислениях и округлении.

\newpage

\section*{Приложение 1. Исходный код проекта}

Исходные коды проекта доступны на Github: https://github.com/webconn/cmc\_MatrixMethods

Ниже приведены исходные коды модулей, не описанных выше.

\lstinputlisting[caption={Исходный код интерфейса программы (файл main.c)}]{../main.c}

\lstinputlisting[caption={Исходный код библиотеки векторных операций (файл matrix.c)}]{../matrix.c}

\lstinputlisting[caption={Исходный код для генерации матрицы по формуле 1 (файл input1.c)}]{../input1.c}

\lstinputlisting[caption={Исходный код для генерации матрицы по формуле 2 (файл input2.c)}]{../input2.c}

\end{document}          